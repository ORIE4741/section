\documentclass{beamer}

\input defs.tex
\input formatting.tex
\newcommand{\citet}[1]{{\footnotesize{\textmd{[#1]}}}}

%\mode<handout>
\mode<presentation>
{
\usetheme{default}
}
\setbeamertemplate{footline}[frame number]

\usepackage{amsmath,amsthm,comment,mathtools}
\usepackage{array,graphicx,amssymb,psfrag}
\usepackage{caption,subcaption}
\usepackage{algorithm,algorithmicx,algpseudocode}
\usepackage{epstopdf}
\epstopdfsetup{update}
\usepackage{tikz}
\usetikzlibrary{shapes}
\usepackage{pgfplots}
\pgfplotsset{compat=newest}
\pgfplotsset{every axis legend/.append style={%
cells={anchor=west}}
}
\usetikzlibrary{arrows}
\tikzset{>=stealth'}
\pgfplotsset{width=12cm}

\bibliographystyle{alpha}

\title{ORIE 4741: Linear Algebra and Gradient Descent}

\date{\textcolor{blue}{September 30, 2019}}
\author{Chengrun Yang \\
	[1ex]
%	Operations Research and Information Engineering \\
}

\begin{document}

\begin{frame}
%\includegraphics[width=.12\linewidth]{cornell_seal.pdf}
\titlepage
\end{frame}


 \begin{frame}{Outline}
 \tableofcontents
 \end{frame}

\section{Full rank matrices}

\begin{frame}{Full rank matrices}
Claim: The followings are equivalent for any matrix $A \in \mathbb{R}^{m \times n}$:
\ben
\item ($Ax=0 \Leftrightarrow x=0$) 
\item $A$ has full column rank 
\item $A^\top A$ is invertible
\een
\vfill
\pause
\textbf{Equivalence of 1 and 2}:
\begin{proof}
	Write $A$ as the concatenation of column vectors $(a_1, a_2, \cdots, a_n)$. $Ax=0$ can then be written as $\sum_{i=1}^n a_i x_i=0$. Thus ($Ax=0 \Leftrightarrow x=0$) is equivalent to the columns of $A$ being linearly independent, i.e. $A$ has full column rank.
\end{proof}
\pause
\textbf{Question: equivalence of 1, 2 and 3?}
\end{frame}

\section{Pseudoinverse}
\begin{frame}{Pseudoinverse}
Definition: for any matrix $A \in \mathbb{R}^{m \times n}$, a pseudoinverse of $A$ is defined as a matrix $A^\dagger \in \mathbb{R}^{n \times m}$ if it satisfies all the following:
\bit
\item $A A^\dagger A = A$
\item $A^\dagger A A^\dagger = A^\dagger$
\item $(A A^\dagger)^\top = A A^\dagger$
\item $(A^\dagger A)^\top = A^\dagger A$

\eit

\pause
\vfill
If $A$ has full column rank, $A^\dagger = (A^\top A)^{-1} A^\top$. Thus:
\bit
\item $A^\dagger A = I_n$
\item $A A^\dagger \neq I_m$ (not necessarily equal)
\eit
But \dots

\pause
\vfill


Claim: If $y \in \text{range}(A)$, then $A A^\dagger y = y$.

\end{frame}

\section{Gradient descent for least squares problem}
\begin{frame}{Convexity}
Definition: A function $f: \mathbb{R}^n \rightarrow \mathbb{R}$ being convex if the domain of $f$ (denoted as $\textbf{dom}(f)$) is a convex set and $\forall x, y \in \textbf{dom}(f)$ and $\theta \in [0,1]$, $f(\theta x + (1- \theta)y) \leq \theta f(x) + (1- \theta)f(y)$.
\vfill
\pause
Equivalent definitions:
\bit
\item (First-order Convexity Condition) Suppose a function $f: \mathbb{R}^n \rightarrow \mathbb{R}$ is differentiable in $\textbf{dom}(f)$. Then $f$ is convex if and only if $\textbf{dom}(f)$ is convex and $\forall x, y \in \textbf{dom}(f)$, $f(y) \geq f(x) + \nabla f(x)^{T}(y-x)$.
\item (Second-order Convexity Condition) Suppose a function $f: \mathbb{R}^n \rightarrow \mathbb{R}$ is twice differentiable in $\textbf{dom}(f)$. Then $f$ is convex if and only if $\textbf{dom}(f)$ is convex, $\forall x \in \textbf{dom}(f)$, $\nabla^2 f \succeq 0$ (positive semi-definite).
\eit
\end{frame}

\begin{frame}{Convergence rate of smooth functions}
A function $f$ is smooth if and only if $\forall x, y \in \textbf{dom}(f)$, $f(y) \leq f(x) + \nabla f(x)^{T}(y-x) + \frac{\beta}{2}||x-y||^2$.

\vfill

\pause
Theorem: Under the following conditions:
\ben
	\item $f: \mathbb{R}^n \rightarrow \mathbb{R}$ is convex and differentiable with $\textbf{dom}(f) = \mathbb{R}^n$
	\item $f$ is smooth with parameter $\beta >0$
	\item  Optimal value $p^{*}=\inf_x f(x)$ is finite and is attained at $x^{*}$
\een
If we perform gradient descent updates $x^{(k+1)} = x - t \nabla f(x^{(k)}) $ on $f$ with a constant step size $t$ that satisfies $0 < t \leq \frac{1}{\beta}$, the number of steps taken to achieve $f(x^{(k)})-p^* \leq \epsilon$ is $O(\frac{1}{\epsilon})$.

\end{frame}

\begin{frame}{Convergence on least squares problem}
Our problem: $\text{minimize} ||y-Xw||^2$

\vfill
First and second-order derivatives: 
$\nabla_w ||y-Xw||^2 = 2X^\top (Xw-y)$

$\nabla^2_w ||y-Xw||^2 = 2X^\top X$

\vfill
Properties of the least squares problem:
\bit
\item Convexity: $\nabla^2_w ||y-Xw||^2 \succeq 0$
\item Smoothness: $||\nabla_w ||y-Xw_1||^2 - \nabla_w ||y-Xw_2||^2|| \leq 2||X^\top X||_\text{op} ||w_1 - w_2||_2$
\eit

Thus if step size $t$ satisfies $0 \leq t \leq \frac{1}{2||X^\top X||_\text{op}}$, we can get a convergence rate of $O(\frac{1}{k})$ with respect to the number of steps $k$.

\end{frame}
%\input{results}


\end{document}
